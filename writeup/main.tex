\documentclass[letterpaper]{article}

\usepackage{aaai}
\usepackage{times}
\usepackage{helvet}
\usepackage{courier}

\frenchspacing
\setlength{\pdfpagewidth}{8.5in}
\setlength{\pdfpageheight}{11in}

% \usepackage[numbers]{natbib}            % Bibliography
\usepackage{amsmath}                    % For Math Proofs
\usepackage{amsthm}                     % For Math Proofs
\usepackage{amssymb}                    % For Math Proofs
\usepackage{amsfonts}                   % For Math Proofs
\usepackage{algorithm}                  % For Algorithms
\usepackage{algorithmic}                % For Algorithms
\usepackage{graphicx}                   % For Graphics/Figures
\usepackage{float}                      % For Graphics/Figures
\usepackage{epstopdf}                   % Convert eps images to pdf
\usepackage{mathabx}                    % For \bigtimes -- Cartesian product

% Create some formatting and definitions.
\theoremstyle{definition}
\newtheorem{problem}{Problem}
\newtheorem{assumption}{Assumption}
\newtheorem{definition}{Definition}
\newtheorem{notation}{Notation}
\newtheorem{theorem}{Theorem}
\newtheorem{corollary}{Corollary}
\newtheorem{remark}{Remark}
\newtheorem{proposition}{Proposition}
\newtheorem{lemma}{Lemma}
\DeclareMathOperator*{\argmin}{argmin}
\DeclareMathOperator*{\argmax}{argmax}
\DeclareMathOperator*{\lvmax}{lvmax}
\DeclareMathOperator*{\leximax}{leximax}
\DeclareMathOperator*{\leximin}{leximin}


\pdfinfo{
/Title (Multi-Objective MDPs with Lexicographic Reward Preferences)
/Author (Kyle Hollins Wray, Shlomo Zilberstein, Abdel-Illah Mouaddib)}

\setcounter{secnumdepth}{0}

\title{Multi-Objective MDPs with Lexicographic Reward Preferences}

\author{Kyle Hollins Wray \and Shlomo Zilberstein \\
  University of Massachusetts \\
  Amherst, MA 01003, USA \\
  {\tt \{wray, shlomo\}@cs.umass.edu} \And
  Abdel-Illah Mouaddib \\
  GREYC - Univeaite de Caen \\
  Bd Marechal Juin, BP 5186 \\
  F14032 Caen Cedex, France \\
  {\tt mouaddib@info.unicaen.fr}}


\begin{document}

\maketitle

% \begin{abstract}
%\begin{quote}
Sequential decision problems that involve multiple objectives are prevalent.  Consider for example a driver of a semi-autonomous car who may want to minimize both travel time and the effort associated with manual driving.  We introduce a rich model called lexicographic MDP (LMDP) and a corresponding planning algorithm called LVI that generalize previous work by allowing for conditional lexicographic preferences with slack.  We analyze the convergence characteristics of LVI and establish its game theoretic properties. The performance of LVI in practice is tested within a realistic benchmark problem in the domain of semi-autonomous driving. Finally, we demonstrate how GPU-based optimization can improve the scalability of LVI and other value iteration algorithms for MDPs.
%\end{quote}
\end{abstract}



\section{Introduction}
\label{sec:introduction}

This document describes work-in-progress for lexicographic value iteration in MOMDPs. Essentially, the idea is that for each state, $\lvmax$ starts with all actions available for the first value function. It then computes the optimal actions to take, but also keeps actions which yield values that are close to optimal (within some tolerance $\delta_1$). Then, it uses this restricted action set to compute the optimal value for the next value function (which has a tolerance $\delta_2$). This continues until the end, when a collection of actions remains after this pruning. The final action is selected from this set by going through the value functions one more time, but this time selecting only the optimal action (or set of actions if a tie occurs) until one remains. If after all this there is \emph{still} a tie, we will just assume a simple preference ordering over actions breaks the final tie.

Note that if all $\delta_i = 0$, it reduces to a pure $\argmax$ selection at each pruning step. This is our original lexicographic preference idea.


\section{Problem Definition}
\label{sec:problem_definition}

A Multi-Objective Markov Decision Process (MOMDP) is a sequential decision process in which an agent controls a domain with a finite set of states. The actions the agent can perform in each state cause a stochastic transition to a successor state. This transition results in a reward, which consistes of a vector of values, each of which depends on the state transition and action. The process unfolds over a finite or infinite number of discrete time steps.  In a standard MDP, there is a single reward function and the goal is to maximize the expected cummulative discounted reward over the sequence of stages. MOMDPs present a more general model with multiple reward functions. We define below a variant of MOMDPs that we call Lexicographic MDP (LMDP), which extends MOMDPs with lexicographic preferences to also include conditional preferences and slack. We then introduce a Lexicographic Value Iteration (LVI) algorithm (Algorithm~\ref{alg:lvi}) which solves LMDPs.

\begin{definition}
    \label{def:lmdp}
    A \textbf{Lexicographic Markov Decision Process (LMDP)} is a represented by a 7-tuple $\langle S, A, T, \mathbf{R}, \delta, \mathcal{S}, o \rangle$:
    \begin{itemize}
        \item $S$ is a finite set of $n$ states, with initial state $s_0 \in S$
        \item $A$ is a finite set of $m$ actions
        \item $T : S \times A \times S \rightarrow [0, 1]$ is a state transition function which specifies the probability of transitioning from a state $s \in S$ to state $s' \in S$, given action $a \in A$ was performed; equivalently, $T(s, a, s') = Pr(s' | s, a)$
        \item $\mathbf{R} = [R_1, \ldots, R_k]^T$ is a vector of reward functions such that $\forall i \in K = \{1, \ldots, k\}$, $R_i : S \times A \times S \rightarrow \mathbb{R}$; each specifies the reward for being in a state $s \in S$, performing action $a \in A$, and transitioning to a state $s' \in S$, often written as $\mathbf{R}(s, a, s') = [R_1(s, a, s'), \ldots, R_k(s, a, s')]$
        \item $\delta = \langle \delta_1, \ldots, \delta_k \rangle$ is a tuple of slack variables such that $\forall i \in K$, $\delta_i \geq 0$
        \item $\mathcal{S} = \{S_1, \dots, S_\ell\}$ is a set which forms an $\ell$-partition over the state space $S$
        \item $o = \langle o_1, \ldots, o_\ell \rangle$ is a tuple of strict preference orderings such that $L = \{1, \ldots, \ell\}$, $\forall j \in L$, $o_j$ is a $k$-tuple ordering elements of $K$
    \end{itemize}
\end{definition}

We consider \emph{infinite horizon} LMDPs (i.e., $h = \infty$), with a \emph{discount factor} $\gamma \in [0, 1)$, due to space considerations. The finite horizon case follows in the natural way. A \emph{policy} $\pi : S \rightarrow A$ maps each state $s \in S$ to an action $a \in A$.

%We call $h \in \mathbb{N} \cup \{ \infty \}$ the \emph{horizon}, i.e., number of steps until the process terminates. The process may be defined as finite ($h < \infty$) or infinite ($h = \infty$). Over the iterations, rewards are discounted by \emph{discount factor} $\gamma$. For finite horizon problems, typically $\gamma = 1$; for infinite horizon problems, $\gamma \in [0, 1)$.
%A \emph{policy} $\pi : S \rightarrow A$ is a mapping from each state $s \in S$ to an action $a \in A$. In finite horizon problems, we have a sequence of policies $\langle \pi_1, \ldots, \pi_h \rangle$ representing a policy for each stage.

Let $\mathbf{V} = [V_1, \ldots, V_k]^T$ be a set of \emph{value functions}. Let each function $V_i^\pi : S \rightarrow \mathbb{R}$, $\forall i \in K$, represent the value of states $S$ following policy $\pi$. The stochastic process of MDPs enable us to represent this using the expected value over the reward for following the policy at each stage.
\begin{equation*}
    \mathbf{V}^\pi(s) = \mathbb{E} \Big[ \sum_{t=0}^\infty \gamma^t \mathbf{R}(s^t, \pi(s^t), s^{t+1}) \Big| s^0 = s, \pi \Big]
\end{equation*}

This allows us to recursively write the value of the state $s \in S$, given a particular policy $\pi$, in the following manner.
\begin{equation*}
    \mathbf{V}^\pi(s) = \sum_{s' \in S} T(s, \pi(s), s') (\mathbf{R}(s, \pi(s), s') + \gamma \mathbf{V}^\pi(s'))
\end{equation*}


\begin{algorithm}[t]
    \begin{algorithmic}[1]
        \caption{Lexicographic Value Iteration (LVI)}
        \label{alg:lvi}
        % \REQUIRE $???$: ???
        \STATE $V \leftarrow 0$
        \STATE $V' \leftarrow 0$
        \WHILE{$\|V - V'\|_\infty^{S} > \epsilon \frac{1 - \gamma}{\gamma}$}
            \STATE $V' \leftarrow V$
            \STATE $V^{fixed} \leftarrow V$
            \FOR{$j = 1, \ldots, \ell$}
                \FOR{$i = o_j(1), \ldots, o_j(k)$}
                    \WHILE{$\| V_i' - V_i \|_\infty^{\mathcal{S}_j} > \epsilon \frac{1 - \gamma}{\gamma}$}
                        \STATE $V_i'(s) \leftarrow V_i(s)$, $\forall s \in S_j$
                        \STATE $V_i(s) \leftarrow B_i V_i'(s)$, $\forall s \in S_j$
                    \ENDWHILE
                \ENDFOR
            \ENDFOR
        \ENDWHILE
        \RETURN $V'$
    \end{algorithmic}
\end{algorithm}


\subsection{Lexicographic Value Iteration}
\label{sec:lvi}

LMDPs lexicographically maximize $V_{o_j(i)}(s)$ over $V_{o_j(i+1)}(s)$, for all $i \in \{1, \ldots, k - 1\}$, $j \in L$, and $s \in S$, using $V_{oj{i+1}}$ to break ties. The model allows for slack as defined by $\eta_{o_j(i)} \geq 0$ (deviation from optimal for a single action change) and $\delta_{o_j(i)} \geq 0$ (deviation from the overall optimal value). As we show below, the classical value iteration algorithm~\cite{Bellman57} can be easily modified to solve MOMDPs with this preference characterization.

%% [[seems overly formal giving a number...]]
For the sake of readability, we use the following convention: Always assume that the ordering is present, unless otherwise stated. 
%% This notation vastly improves readability.
This allows us to omit the explicit ordering $o_j(\cdot)$ for subscripts, sets, etc. For example, $V_{i+1} \equiv V_{o_j(i+1)}$, and $\{1, \ldots, i - 1\}$ $\equiv$ $\{o_j(1), \ldots, o_j(i - 1)\}$.


%The algorithm begins by running value iteration for $V_1$ until convergence. This determines the subset of actions available to the next value function at each state. Then, value iteration is run on $V_2$ until convergence, which defines further restricted subset of actions for $V_3$. This process iterates until $V_k$ converges, at which point we have our final policy.

First, Equation~\ref{eq:value_of_state_action} defines $Q_i(s, a)$, the value of taking an action $a \in A$ in a state $s \in S$ according to objective $i \in K$.
\begin{equation}
    \label{eq:value_of_state_action}
    Q_i(s, a) = \sum_{s' \in S} T(s, a, s') (R_i(s, a, s') + \gamma V_i(s'))
\end{equation}

With this definition in place, we may define the aforementioned restricted set of actions for each state $s \in S$. For $i = 1$, let $A_1(s) = A$ and for all $i \in \{1, \ldots, k - 1\}$ let $A_{i+1}(s)$ be defined following Equation~\ref{eq:restricted_actions_set}.
\begin{equation}
    \label{eq:restricted_actions_set}
    A_{i+1}(s) = \{ a \in A_i(s) | \max_{a' \in A_i(s)} Q_i(s, a') - Q_i(s, a) \leq \eta_i \}
\end{equation}
For reasons explained in Section~\ref{sec:theoretical_analysis}, we let $\eta_i = (1 - \gamma) \delta_i$.

Finally, let Equation~\ref{eq:lvi_slack} below be the \emph{Bellman update equation} for MOMDPs with lexicographic reward preferences for $i \in K$, using slack $\delta_i \geq 0$, for all states $s \in S$. If $i > 1$, then we require $V_{i-1}$ to have converged for all states.
\begin{equation}
    \label{eq:lvi_slack}
    V_i(s) = \max_{a \in A_i(s)} Q_i(s, a)
\end{equation}

Within the algorithm, we leverage a modified value iteration with slack Bellman update equation (from Equation~\ref{eq:lvi_slack}) denoted as $B_i$. We either use $V_i$ for $s \in S_j \subseteq S$ or $V_i^{fixed}(s)$ for $s \in S \setminus S_j$, as shown in Equation~\ref{eq:lvi} below, with $[\cdot]$ denoting Iverson brackets.
\begin{align}
    B_i V_i'(s) &= \max_{a \in A_i(s)} \sum_{s' \in S} T(s, a, s') (R_i(s, a, s') + \gamma \bar{V}_i(s')) \label{eq:lvi} \\
    \bar{V}_i(s') &= V_i'(s') [s \in S_j] + V_i^{fixed}(s') [s \notin S_j] \label{eq:lvi_V_bar}
\end{align}
Also, in each infinity norm we denote the domain of the maximization such that $\| \cdot \|_\infty^Z = \max_{z \in Z} | \cdot |$.



\section{Theoretical Analysis}
\label{sec:theoretical_analysis}

We provide a proof of convergence for $\lvmax$ value iteration in Proposition~\ref{prop:lvmax_convergence}.

\begin{proposition}
    \label{prop:lvmax_convergence}
    Lexicographic vector max ($\lvmax$) value iteration converges to a unique fixed point of value functions, given discount factors $\gamma_i \in [0, 1)$ for all $i \in K$ and \emph{Lipschitz constant} $\nu \in [0, 1)$ defined as
    \begin{equation}
        \label{eq:lvmax_convergece_nu}
        \nu = \max_{i \in K} \Big( \gamma_i + (1 - \gamma_i) \frac{\delta_i}{|R_i^+ - R_i^-|} \Big)
    \end{equation}
    with $R_i^+ = \max_{s \in S} \max_{a \in A} \max_{s' \in S} R_i(s, a, s')$ and $R_i^- = \min_{s \in S} \min_{a \in A} \min_{s' \in S} R_i(s, a, s')$.
\end{proposition}

\begin{proof}
To prove convergence, we must show that the Bellman optimality equation converges to a unique fixed point. We will do this by induction on $i \in K$. First, we must state some definitions.

For a metric space $\langle X, d \rangle$, where $X$ is a set and $d$ is a distance metric, a map $f : X \rightarrow X$ is called a \emph{contraction map} if there exists an $\alpha$ such that $d(f(x), f(y)) = \alpha d(x, y)$, for all $x, y \in X$.

Let the space $X_i$ be the \emph{space of value functions for $i \in K$}, i.e., we have $V_i = [V_i(s_1), \ldots, V_i(s_n)]^T \in X_i$. Let the distance metric $d_i$ be the \emph{max norm}, i.e., $\|V_i\|_\infty = \max_{s \in S} |V_i(s)|$. Since $\gamma_i \in [0, 1)$, the metric space $M_i = \langle X_i, d_i \rangle$ is a \emph{complete metric space}; every Cauchy sequence of $M_i$ converges to a point in $M_i$.

Let the $\lvmax$ Bellman optimality equation's (Equation~\ref{eq:lvmax_value_iteration}) be defined as an operator $B$, i.e., $V^{t+1} = B V^t$, for $V^t, V^{t+1} \in X$ with $t \geq 0$. Let element $i \in K$ of the operator be $B_i$, such that $V_i^{t+1} = (B V^t)_i = B_i V_i^t$. We prove the operator $B_i$ is a contraction map in $M_i$ for all $i \in K$, given either that $i=1$ or that the previous $i-1$ has converged to within $\epsilon$ of its fixed point.

Let $V_{1i}, V_{2i} \in X_i$ be any two value function vectors, and $\gamma_i \in [0, 1)$. We first apply the definition of $\lvmax$ from Equation~\ref{eq:lvmax}.
\begin{equation*}
    \| B_i V_{1i} - B_i V_{2i} \|_\infty = \max_{s \in S} | \max_{a \in A_{1i}^*} Q_{1i}(s, a) - \max_{a \in A_{2i}^*} Q_{2i}(s, a) |
\end{equation*}

By Definition~\ref{def:lvmax}, $A_{1i}^* \subseteq A_{1i}$. Therefore, for all $s \in S$, $\max_{a \in A_{1i}^*} Q_{1i}(s, a) \leq \max_{a' \in A_{1i}} Q_{1i}(s, a)$. Also by Definition~\ref{def:lvmax}, for all $a^* \in A_{2i}^* \subseteq A_{2,i+1} = \{a \in A_{2i} | \max_{a' \in A_{2i}} Q_{2i}(s, a') - Q_{2i}(s, a) \leq \delta_i \}$. Thus,
\begin{align*}
    \max_{a' \in A_{2i}} Q_{2i}(s, a') - \delta_i &\leq Q_{2i}(s, a^*) \leq \max_{a' \in A_{2i}^*} Q_{2i} (s, a') \\
    -\max_{a' \in A_{2i}^*} Q_{2i} (s, a') &\leq \delta_i - \max_{a' \in A_{2i}} Q_{2i}(s, a')
\end{align*}

Without loss of generality, assume that $B_i V_{1i} \geq B_i V_{2i}$, making the absolute value optional in the first equation below. If the opposite is true, then we may simply switch the logic for deriving the two above upper bounds and apply those instead throughout.

Combine these three facts, and we obtain the following.
\begin{align*}
    &\| B_i V_{1i} - B_i V_{2i} \|_\infty \\
    &\leq \max_{s \in S} | \max_{a \in A_{1i}} Q_{1i}(s, a) - \max_{a \in A_{2i}} Q_{2i}(s, a) + \delta_i | \\
    &\leq \max_{s \in S} | \max_{a \in A_{1i}} Q_{1i}(s, a) - \max_{a \in A_{2i}} Q_{2i}(s, a)| + |\delta_i|
\end{align*}

By Definition~\ref{def:lvmax}, when $i=1$ we have $A_{1i} = A_{2i} = A$. Similarly, when $i \in \{2, \ldots, k\}$ given that $i-1$ has converged to within $\epsilon$ of its fixed point, it yields a unique fixed set of actions $A'$, with $A_{1i} = A_{2i} = A' \subseteq A$. Let us denote this fixed actions set as $\bar{A}_i$ for all $i \in K$. Also, as part of the $Q(\cdot)$ values, we distribute $T(\cdot)$ to each $R(\cdot)$ and $V(\cdot)$ in the summations, then apply the property: $\max_x f(x) + g(x) \leq \max_x f(x) + \max_x g(x)$, twice.
\begin{align*}
    &\| B_i V_{1i} - B_i V_{2i} \|_\infty \\
    &\leq \max_{s \in S} \Big| \max_{a \in \bar{A}_i} \Big( \sum_{s' \in S} T(s, a, s') R_i(s, a, s') \\
    &\quad \quad + \gamma_i \sum_{s' \in S} T(s, a, s') V_{1i}(s') \Big) \\
    &\quad \quad - \max_{a \in \bar{A}_i} \Big( \sum_{s' \in S} T(s, a, s') R_i(s, a, s') \\
    &\quad \quad - \gamma_i \sum_{s' \in S} T(s, a, s') V_{2i}(s') \Big) \Big| + \delta_i \\
    &\leq \max_{s \in S} \Big| \max_{a \in \bar{A}_i} \sum_{s' \in S} T(s, a, s') R_i(s, a, s') \\
    &\quad \quad + \gamma_i \max_{a \in \bar{A}_i} \sum_{s' \in S} T(s, a, s') V_{1i}(s') \\
    &\quad \quad - \max_{a \in \bar{A}_i} \sum_{s' \in S} T(s, a, s') R_i(s, a, s') \\
    &\quad \quad - \gamma_i \max_{a \in \bar{A}_i} \sum_{s' \in S} T(s, a, s') V_{2i}(s') \Big| + \delta_i \\
    &\leq \max_{s \in S} \Big| \gamma_i \max_{a \in \bar{A}_i} \sum_{s' \in S} T(s, a, s') V_{1i}(s') \\
    &\quad \quad - \gamma_i \max_{a \in \bar{A}_i} \sum_{s' \in S} T(s, a, s') V_{2i}(s') \Big| + \delta_i
\end{align*}
Note that we can pull out $\gamma_i \in [0, 1)$. Recall, that for any two functions $f$ and $g$, $| \max_x f(x) - \max_x g(x) | \leq \max_x | f(x) - g(x) |$.
\begin{align*}
    &\| B_i V_{1i} - B_i V_{2i} \|_\infty \\
    &\leq \gamma_i \max_{s \in S} \max_{a \in \bar{A}_i} \Big| \sum_{s' \in S} T(s, a, s') (V_{1i}(s') - V_{2i}(s')) \Big| + \delta_i
\end{align*}

Since $\sum_{s' \in S} T(s, a, s') = 1$ and for all $s' \in S$, $T(s, a, s') \in [0, 1]$, it is defined on the $n$-simplex. It then scales the vertices by the values $R(\cdot)$ or $V(\cdot)$. This forms simple convex polytope. Convex polytopes obtain their maximum value at the vertices (or on an entire edge or face, which includes the vertices). Therefore, we may exclusively maximize over these vertices ($R(\cdot)$ and $V(\cdot)$), and may simply drop both maximizations which select the weights (i.e., maximization over $s \in S$ and $a \in A$).
\begin{align*}
    &\| B_i V_{1i} - B_i V_{2i} \|_\infty \leq \gamma_i \max_{s' \in S} \Big| V_{1i}(s') - V_{2i}(s') \Big| + \delta_i \\
    &\leq \gamma_i \| V_{1i} - V_{2i} \|_\infty + \delta_i \frac{\| V_{1i} - V_{2i} \|_\infty}{\| V_{1i} - V_{2i} \|_\infty} \\
    &\leq \Big( \gamma_i + \frac{\delta_i}{\| V_{1i} - V_{2i} \|_\infty} \Big) \| V_{1i} - V_{2i} \|_\infty
\end{align*}

We can place a constant upper bound on this value to remove the denominator $\| V_{1i} - V_{2i} \|_\infty$. Consider any finite or finite sequence of states and actions performed by the agent: $z = \langle s^0, a^0, s^1, a^1, \ldots \rangle$. The utility $u_i^h(z)$ at horizon $h \in \mathbb{N} \cup \{ \infty \}$ is bounded:
\begin{align*}
    u_i^h(\langle s^0, a^0, s^1, a^1, \ldots \rangle) &= \sum_{t=0}^h \gamma_i^t R_i(s^t, a^t, s^{t+1}) \\
        &\leq \sum_{t=0}^\infty \gamma_i^t R_i^+ \leq \frac{R_i^+}{1 - \gamma_i}
\end{align*}
Similarly,
\begin{equation*}
    -u_i^h(\langle s^0, a^0, s^1, a^1, \ldots \rangle) \geq \frac{R_i^-}{1 - \gamma_i}
\end{equation*}
The value function is therefore bounded by these values, because including state transitions would only decrease the values.
\begin{align*}
    \| V_{1i} - V_{2i} \|_\infty &\leq \Big| \frac{R_i^+}{1 - \gamma_i} - \frac{R_i^-}{1 - \gamma_i} \Big| \leq \frac{|R_i^+ - R_i^-|}{1 - \gamma_i} \\
    \Rightarrow \quad \quad \frac{-1}{\| V_{1i} - V_{2i} \|_\infty} &\leq -\frac{1 - \gamma_i}{|R_i^+ - R_i^-|}
\end{align*}

Now we may obtain final result, proving that the Bellman operator $B_i$ is a contraction map on metric space $M_i$, for all $i \in K$.
\begin{align*}
    &\| B_i V_{1i} - B_i V_{2i} \|_\infty \leq \Big( \gamma_i - \delta_i \frac{-1}{\|V_{1i} - V_{2i}\|_\infty}  \Big) \| V_{1i} - V_{2i} \|_\infty \\
    &\leq \Big( \gamma_i - \delta_i \Big( -\frac{1 - \gamma_i}{|R_i^+ - R_i^-|} \Big) \Big) \| V_{1i} - V_{2i} \|_\infty \\
    &\leq \Big( \gamma_i + (1 - \gamma_i) \frac{\delta_i}{|R_i^+ - R_i^-|} \Big) \| V_{1i} - V_{2i} \|_\infty \\
    &\leq \nu \| V_{1i} - V_{2i} \|_\infty
\end{align*}

Thus, for all $i \in K$, by definition of a contraction map, $B_i$ admits at most one fixed point. Additionally, since $M_i$ is a complete metric space, we can guarantee convergence to a unique fixed point. \emph{Banach's fixed point theorem} states that if $M_i = \langle X_i, d_i \rangle$ is a complete metric space and $B_i: X_i \rightarrow X_i$ is a contraction map, then $B_i$ admits a unique fixed point $V_i^* \in X_i$, i.e., $B_i V_i^* = V_i^*$. Thus, we will have convergence of value iteration to a unique fixed point $V_i^* \in X_i$, for all $i \in K$.
\end{proof}

We may also derive an equation which guarantees convergence to within $\epsilon > 0$ of the fixed point, as shown in Proposition~\ref{prop:lvmax_convergence_check}.

\begin{proposition}
    \label{prop:lvmax_convergence_check}
    Lexicographic vector max ($\lvmax$) value iteration converges to within $\epsilon > 0$ of its unique fixed point once $\| V^{t+1} - V^t \|_\infty < \epsilon \frac{1 - \nu}{\nu}$.
\end{proposition}

\begin{proof}
From Proposition~\ref{prop:lvmax_convergence}, for all $i \in K$, a corollary of Banach's fixed point theorem is that the speed of convergence to within $\epsilon > 0$ of the fixed point $x^*$ is known (using the generic notation from above for a metric space).
\begin{align*}
    d(x^*, x_{t+1}) &\leq \frac{\alpha}{1 - \alpha} d(x_{t+1}, x_t) \\
    \| V_i^* - V_i^{t+1}\|_\infty &\leq \frac{\nu}{1 - \nu} \| V_i^{t+1} - V_i^t \|_\infty
\end{align*}

Since we want the distance from the fixed point $V_i^* \in X_i$ to be $\epsilon$, we may rewrite the equation accordingly.
\begin{align*}
    \epsilon &\leq \frac{\gamma_i}{1 - \gamma_i} \| V_i^{t+1} - V_i^t \|_\infty \\
    \epsilon \frac{1 - \gamma_i}{\gamma_i} &\leq \| V_i^{t+1} - V_i^t \|_\infty
\end{align*}
The above equation states that we are at least $\epsilon$ (or more) away from the fixed point when the maximum difference (over the states) between iterations satisfies the inequality. Therefore, we flip the inequality to create a convergence criterion, which ensures that we are $\epsilon$ or less from the fixed point. Finally, this must be true for all $i \in K$, so we may rewrite it with the infinity norm defined over both $K$ and $S$.
\begin{equation*}
    \quad \| V^{t+1} - V^t \|_\infty < \epsilon \frac{1 - \nu}{\nu}
\end{equation*}
\end{proof}

It is straightforward to show that $\lvmax$ value iteration generalizes value iteration, as shown in Proposition~\ref{prop:lvmax_generalizes_vi}.
\begin{proposition}
    \label{prop:lvmax_generalizes_vi}
    Lexicographic vector max ($\lvmax$) value iteration generalizes value iteration.
\end{proposition}

\begin{proof}
Let $M = \langle S, A, T, \mathbf{R} \rangle$ with $\mathbf{R} = \langle R_1 \rangle$ and $\delta_1 = 0$. By Definition~\ref{def:lvmax}, for all $s \in S$, $\mathbf{V}(s) = \lvmax_{a \in A} \mathbf{Q}(s, a) = \max_{a \in A_1^*} Q_1(s, a)$. Since $A_1^* = A_k = A_1 = A$, we have $V_1(s) = \max_{a \in A} Q_1(s, a)$. This is value iteration on $M' = \langle S, A, T, R_1 \rangle$.
\end{proof}

Typically, a multi-objective optimization problem converts the problem into a objective function by summing the value function and weighting them in a particular manner. Our $\lvmax$ value iteration returns a different solution from the linearly weighted function approaches used to solve MOMDPs. Proposition~\ref{prop:lvmax_different_solution} proves this fact.

% Let the set of all policies returned by value iteration operator $B$ be denoted as $\Pi_B$. Formally, we say that the value iteration operator $B$ is \emph{unique} with respect to value iteration operator $B'$, if and only for some $\epsilon > 0$ and $\gamma_i \in [0, 1)$ in MDP $M = \langle S, A, T, R \rangle$, $\Pi_B \not\subseteq \Pi_{B'}$ and $\Pi_{B'} \not\subseteq \Pi_B$.

% \begin{proposition}
%     \label{prop:lvmax_uniqueness}
%     For $\epsilon > 0$ and $\gamma \in [0, 1)$, let $\Pi_{vi}$ be the set of all policies returned by value iteration operator $B_{vi}$ using a weighted objective function with weights $\mathbf{w} \in \mathbb{R}^k$. Also, let $\Pi_{lv}$ be the set of all policies returned by value iteration operator $B_{lv}$ as defined by Equation~\ref{eq:lvmax_value_iteration} with $\epsilon$ and $\gamma$. $B_{lv}$ is unique with respect to $B_{vi}$.
% \end{proposition}

% \begin{proof}
% By the definition of uniqueness, must show that there exists an MDP $M = \langle S, A, T, R \rangle$ such that $\Pi_{vi} \not\subseteq \Pi_{lv}$ and $\Pi_{lv} \not\subseteq \Pi_{vi}$. Assume by contradiction that $\forall M$, $\Pi_{vi} \subseteq \Pi_{lv}$ or $\Pi_{lv} \subset \Pi_{vi}$.

% \end{proof}


\begin{proposition}
    \label{prop:lvmax_different_solution}
    Let the normal Bellman's equation, with linear weights $\mathbf{w} \in \bigtriangleup^k$, and $\lvmax$'s value iteration's resulting value functions be denoted as the $n$-by-$k$ matrices $V^*$ and $V_{lv}^*$, respectively. There exist a class of MOMDPs $\mathcal{M}$ such that $V^* \neq V_{lv}^*$.
\end{proposition}

\begin{proof}
Assume by contradiction that for all $M \in \mathcal{M}$, $V^* = V_{lv}^*$, i.e., there always exists a set of weights $\mathbf{w} \in \bigtriangleup^k$ which makes this so. We know three things for all $s \in S$: $V_{\mathbf{w}}^*(s) = B V_{\mathbf{w}}^*(s)$, $V_{\mathbf{w}}^*(s) = \mathbf{w} \mathbf{V}^*(s)$, and $\mathbf{V}_{lv}^*(s) = B_{lv} \mathbf{V}_{lv}^*(s)$. Therefore, we have:
\begin{align*}
    f(\mathbf{V}_{lv}^*(s), \mathbf{w}) &= f(\mathbf{V}^*(s), \mathbf{w}) = V_{\mathbf{w}}^*(s) = B V_{\mathbf{w}}^*(s) \\
        &= B( f(\mathbf{V}^*(s), \mathbf{w})) = B (f(B_{lv} \mathbf{V}_{lv}^*(s), \mathbf{w}))
\end{align*}
We now apply the Bellman optimality operator to the right side, as well as the $\lvmax$ operator.
\begin{align*}
    f(\mathbf{V}_{lv}^*(s), \mathbf{w}) &= \max_{a \in A} \Big( \sum_{s' \in S} T(s, a, s') f(\mathbf{R}(s, a, s'), \mathbf{w}) \\
        &+ \gamma \sum_{s' \in S} T(s, a, s') f(\max_{a' \in A_i^*} \mathbf{Q}_{lv}^*(s', a'), \mathbf{w}) \Big)
\end{align*}
Apply the properties the linearity of $f$.
\begin{align*}
    0 &= \max_{a \in A} f\Big( \sum_{s' \in S} T(s, a, s') \mathbf{R}(s, a, s') \\
        &+ \gamma \sum_{s' \in S} T(s, a, s') \max_{a' \in A_i^*} \mathbf{Q}_{lv}^*(s', a') - \mathbf{V}_{lv}^*(s), \mathbf{w} \Big) \\
    0 &= \max_{a \in A} f\Big( \mathbf{x}_a, \mathbf{w} \Big)
\end{align*}
Since $\mathbf{w} \in \bigtriangleup^k$, and $f$ is a linear sum of weights and components, we will show that $\exists M \in \mathcal{M}$ such that $\exists s \in S$ such that $\forall i \in K$, $\mathbf{x}_a > 0$.
\end{proof}


\begin{proposition}
    \label{prop:lvmax_dead_end_avoidance}
    Let $M = \langle S, A, T, \mathbf{R} \rangle$, with $\mathbf{R} = \langle R_1, R_2 \rangle$, be a MOMDP with dead ends for $R_1$ and goal states for $R_2$. Assume there exists a proper policy. With $\gamma = 1$, $\lvmax$ value iteration converges to a policy which strictly avoids dead ends, and otherwise strictly prefers goal states.
\end{proposition}



\section{Experimentation}
\label{sec:experimentation}

This section is just to provide some initial experiments.

\subsection{Grid World with Dead Ends}
First, we implemented a model similar to the classic grid world problem. The agent may move north, south, east, and west in a $w$-by-$h$ grid world. The agent moves successfully with a $0.8$ probability, and fails by moving right or left, each with a $0.1$ probability. At the edges, if the agent cannot move, it remains still.

Throughout the area, dead ends are placed (denoted by ``-'') as well as goal states (denoted by ``+''). In the normal MDP version, dead ends have a reward of $-\infty$, and only have a positive weight of $1.0$ on the transition probability for remaining in the dead end. Goal states have a reward of $1.0$, and all other states have a small penalty of $-0.03$.

In our MOMDP with lexicographic reward function model, we separate the dead ends and goal states into two reward functions. The first reward $R_1$ provides a $0.0$ reward for all states, and a $-1.0$ reward for transitioning to a dead end. The second reward $R_2$ yields a $1.0$ for transitioning to a goal state, and a $-0.03$ for all other states.

We implemented this initial test in Python 2.7. Figures~\ref{fig:seed_1},~\ref{fig:seed_2}, and~\ref{fig:seed_3} show some example output from the model in this domain. Interestingly, it appears to work very well at strictly avoiding dead ends, and then optimizing around the remaining action possibilities.
\begin{figure}[h]
    \centering
    \includegraphics[width=0.4\textwidth,bb=0 0 279 280]{seed_1.png}
    \caption{First example's policy.}
    \label{fig:seed_1}
\end{figure}

\begin{figure}[h]
    \centering
    \includegraphics[width=0.4\textwidth,bb=0 0 561 81]{seed_2.png}
    \caption{Second example's policy.}
    \label{fig:seed_2}
\end{figure}

\begin{figure}[h]
    \centering
    \includegraphics[width=0.4\textwidth,bb=0 0 418 416]{seed_3.png}
    \caption{Third example's policy.}
    \label{fig:seed_3}
\end{figure}

There are additional scenarios for which $\lvmax$ is useful. If there are primary and secondary goals, they can be represented by different rewards, with primary goal states strictly favored over secondary goal states.

\begin{itemize}
    \item If there exists a perfect policy, then it seems to always converge with $\gamma = 1$. The convergent policy always strictly avoids the dead ends, as desired.
    \item Without a perfect policy, it converges but the policy can often be off if a lot of dead ends are close to one another. This is probably one example where the slack variables $\delta_1, \ldots, \delta_k$ could be useful.
    \item These results do not utilize the slack variables: $\delta_1, \ldots, \delta_k$.
\end{itemize}


\subsection{Multi-Objective Autonomous Driving}

To demonstrate the usefulness of both the lexicographic ordering as well as the slack variables, we are considering the autonomous driving domain. We will consider a path planning agent acting over an OpenStreetMap graph, which strictly prefers: shorter time/distance $>$ saving fuel/money $>$ stopping at optional amenities.

Our agent has some slack in optimizing the distance traveled. This allows for it to take an extra 5 minutes to go a bit out of the way in order to save some fuel, or stop at an optional amenity. This same slack holds for the other value functions as well.

We also can model dead ends in this domain. Consider an agent that can choose to drive faster. This might result in a ticket if they drive too fast an get caught, which could be considered a dead end.

This is currently being developed in C++, with supporting scripts to read the OpenStreetMap (OSM) file format written in Python.


\section{Related Work}
\label{sec:related_work}

Recent work by~\citeauthor{Mouaddib04-MultiObjectivePathPlanning} used a strict lexicographic preference ordering for MOMDPs~\cite{Mouaddib04-MultiObjectivePathPlanning}. Their work formed the foundation on which we built our $\lvmax$ value iteration, which generalizes their work through the introduction of slack variables. Other work by~\citeauthor{Perny13-LorenzOptimalSolutionsMOMDPs} used Lorenz dominance with a weighted value functions favor more ``fair'' values~\cite{Perny13-LorenzOptimalSolutionsMOMDPs}.

In the past, others have used lexicographic ordering over value functions, calling this technique \emph{ordinal dynamic programming} \cite{Mitten74-PreferenceOrderDynamicProgramming,Sobel75-OrdinalDynamicProgramming}. Within the context of an MDP, \citeauthor{Mitten74-PreferenceOrderDynamicProgramming} assumed a specific preference ordering over outcomes in the finite horizon case. \citeauthor{Sobel75-OrdinalDynamicProgramming} extended this model to infinite horizon MDPs, but did not fully capture our $\lvmax$ formulation under value iteration, slack variables generalizes their approach, and we present experiments showing its efficacy in an applied setting. Ordinal dynamic programming has also been explored within the reinforcement learning community~\cite{Gabor98-MultiObjectiveReinforcementLearning,Natarajan05-DynamicPreferencesMultiCriteriaRL}, even with a similar notion of slack variables, but has not been represented in the general form we present.

\citeauthor{Barbara88-MaxminLeximax} characterized an operator called $\leximin$ within an economics context, which is closely related to $\lvmax$, except that the ordering is slightly different and does not include slack variables~\cite{Barbara88-MaxminLeximax}. Since its inception, it has enjoyed use outside the domain of MDPs by other economics researchers~\cite{Bossert94-RankingOpportunitySets,Fargier05-QuantitativeDecisionMaking,Arlegi05-FreedomOfChoice}.

Our algorithm can capture the strict avoidance of dead ends as well as loops. Interestingly,~\citeauthor{Kolobov12-TheoryGoalOrientedMDPsDeadEnds} showed that unavoidable dead ends can be represented by a ``price''~\cite{Kolobov12-TheoryGoalOrientedMDPsDeadEnds}. The resulting formulation resembles a specific, non-slack variable version of $\lvmax$ value iteration.

A solid survey of the approaches used to solve MOMDPs is provided by~\citeauthor{Roijers13-SurveyMultiObjective}~\shortcite{Roijers13-SurveyMultiObjective}, but other models exist. Constrained MDPs (CMDPs) are another formulation of MOMDPs, but to our knowledge no one has explored lexicographic preferences over rewards in this domain~\cite{Altman99-CMDPs}. Additionally,~\citeauthor{Gonzales11-DecisionMakingMultipleObjectivesGAINetworks} used Generalized Additive Decomposable (GAI) networks to capture the decomposability of the utility functions to model preferences~\cite{Gonzales11-DecisionMakingMultipleObjectivesGAINetworks}.


% \input{5_experimentation}

% \input{6_discussion}

% \input{7_related_work}

% \input{8_conclusion}

% \input{9_acknowledgements}

\bibliographystyle{aaai}
\bibliography{references}

\end{document}
